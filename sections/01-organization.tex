% -----------------------------------------------
% chktex-file 44
\documentclass[../index.tex]{subfiles}
\begin{document}

% -------------------------------------

\renewcommand{\sectiontitle}{Organization of a basic compiler}
\section{\sectiontitle}

% ---------------------------
\renewcommand{\currenttitle}{Front-end and back-end}
\begin{frame}{\currenttitle}
  Much like a full-stack web application, we can split the compiler into a
  \textbf{front-end} and a \textbf{back-end}: \\[2em]

  \begin{tikzpicture}[%
    wrapper/.style={matrix of nodes, nodes=typetag, row sep=1em},
    container/.style={draw, inner sep=1ex},
    typetag/.style={draw=gray, inner sep=1ex, anchor=west},
    title/.style={draw=none, inner sep=0pt}
  ]
    \matrix[wrapper] (Frontend) {
      |[title]|Front-end \\
      Lexical analysis \\
      Parsing \\
      Semantic analysis \\
      Intermediate code generation \\
    };
    \matrix[wrapper, right=of Frontend.north east, matrix anchor=north west] (Backend) {
      |[title]|Back-end \\
      More optimization \\
      Code generation \\
    };

    \node[container, fit=(Frontend)] {};
    \node[container, fit=(Backend)] {};
  \end{tikzpicture}
\end{frame}

% ---------------------------
\begin{frame}{\currenttitle}
  The front-end:
  \begin{itemize}
    \item Deals with the details of the source language
    \item Verifies that the programmer's code is correct
    \item Generates an intermediate representation that the backend can handle
  \end{itemize}

  The back-end:
  \begin{itemize}
    \item Is ignorant to the details of the source language
    \item Performs optimizations specific to the target machine
    \item Generates the final executable program
  \end{itemize}
\end{frame}


% ---------------------------
\begin{frame}{\currenttitle}
  \onslide<+->{What does this separation mean?} \\[2em]

  \onslide<+->{With \textbf{\texttt{M}} front-ends for \textbf{\texttt{M}}
  source languages\ldots{}}
  \onslide<+->{and \textbf{\texttt{N}} back-ends for \textbf{\texttt{N}}
    target machines,} \\[1em]
    
  \hspace*{2em}
  \onslide<+->{\textit{we can write \textbf{\texttt{M x N}} compilers}}
\end{frame}

% -------------------------------------

\renewcommand{\sectiontitle}{Syntax analysis}
\section{\sectiontitle}

% ---------------------------
\renewcommand{\currenttitle}{Understanding the source code}
\begin{frame}{\currenttitle}
  We want to parse source code into data that we can analyze and transform. \\[2em]

  Typically, compilers first construct an \textbf{abstract syntax tree (AST)}:

  \begin{center}
    \begin{tikzpicture}[
      font=\ttfamily,
      nodes={draw=none},
      level distance=0.8cm,
      ->
    ]
      \node (Code) at (0,-1.5) [inner sep=0.5em, outer sep=2] {
          a = b + c * d("hello");
      };

      \node (AST) at (4,0) {=}
        child {node {a}}
        child {node {+}
          child {node {b}}
          child {node {*}
            child {node {c}}
            child {node {call}
              child {node {d}}
              child {node {"hello"}}
            }
          }
        }
      ;
    \end{tikzpicture}
  \end{center}
\end{frame}

% ---------------------------
\begin{frame}{\currenttitle}
  \onslide<+->{
    It's the job of the \textbf{parser} to convert the \textbf{tokens}
    \texttt{a}, \texttt{=}, \texttt{b}, \texttt{+}, \ldots{} into the syntax
    tree \\[1em]
  }

  \onslide<+->{
    But before that, how do we get those tokens?
  }
\end{frame}

% ---------------------------
\renewcommand{\currenttitle}{Lexical analysis}
\begin{frame}[fragile]{\currenttitle}
  To convert textual source code into \textbf{tokens}, we must perform
  \textbf{lexical analysis}: \\[1em]

  Given the source code: 

  \newcommand{\token}[2]{%
    {
      \scriptsize
      \begin{tabular}{|l |l |}
        \hline
        \textbf{#2} & #1 \\
        \hline
      \end{tabular}
    }
  }

  \begin{lstlisting}[language=C++,xleftmargin=1.5em]
int main() {
  return 1;
}
  \end{lstlisting}

  The \textbf{lexer / lexical analyzer} might spit out the tokens: \\[0.5em]

  \begin{tikzpicture}
    \newcommand{\tokennode}[2]{
      \node [shape=rectangle, anchor=west, outer sep=.0] {
        \token{#1}{#2}
      };
    }

    \matrix[ampersand replacement=\&, matrix of nodes, align=left] {
      \tokennode{int}{Type}   \& \tokennode{main}{Identifier} \& \tokennode{(}{Symbol}        \\
      \tokennode{)}{Symbol}   \& \tokennode{\{}{Symbol}       \& \tokennode{return}{Keyword}  \\
      \tokennode{1}{Literal}  \& \tokennode{;}{Symbol}        \& \tokennode{\}}{Symbol}       \\
    };
  \end{tikzpicture}
\end{frame}

% ---------------------------
\begin{frame}[fragile]{\currenttitle}
  To implement a lexer, we can:

  \begin{itemize}
    \item Manually scan and lookahead the input characters
    \item Implement a \textbf{finite state machine}
  \end{itemize}

  In practice, the latter method typically uses a \textbf{lexer generator} to
  generate the state machine based on rules
\end{frame}


% ---------------------------
\begin{frame}[fragile]{\currenttitle}
  \only<+>{
    To give an idea, a finite state machine to recognize a floating-point:
  }

  \vspace*{1em}

  \newcommand{\statenode}[2]{\node [state] (#1) [#2] {$#1$}}
  \begin{center}
    \begin{tikzpicture}[shorten >= 1pt, node distance=2cm, on grid, auto, font=\ttfamily]
      \tikzstyle{every state}=[fill={rgb:black,1;white,10}]

      \node [state, initial]    (s_0)                       {$s_0$};
      \node [state]             (s_1) [above right of=s_0]  {$s_1$};
      \node [state, accepting]  (s_2) [below right of=s_1]  {$s_2$};
      \node [state, accepting]  (s_3) [right of=s_2]        {$s_3$};

      \path[->]
        (s_0) edge                node {+/-} (s_1)
              edge                node {0-9} (s_2)
        (s_1) edge                node {0-9} (s_2)
        (s_2) edge  [loop below]  node {0-9} ()
              edge                node {.}   (s_3)
        (s_3) edge  [loop below]  node {0-9} ()
        ;
    \end{tikzpicture}
  \end{center}

  \only<+>{
    \begin{itemize}
      \item Start on $s_0$
      \item Read input and move accordingly until the end of the input
        \begin{itemize}
          \item If we end on a double-ringed (accepting) state, the input
            string is a floating-point literal
          \item If not, it's not
          \item If we encounter an input that doesn't have a transition, it's
            not
        \end{itemize}
    \end{itemize}
  }
\end{frame}

% ---------------------------
\begin{frame}[fragile]{\currenttitle}
  To implement a proper lexer using these ideas:

  \begin{itemize}
    \item Combine this with other state machines that recognize other tokens
      (string literals, keywords, etc.) into one large machine
    \item Read input until we come across the end or an input character with no
      transition on the current state, then backtrack until we find the longest
      match (most recent accepting state)
  \end{itemize}
\end{frame}

% ---------------------------
\begin{frame}[fragile]{\currenttitle}
  Additionaly, the lexer may:

  \begin{itemize}
    \item Register identifiers it encounters in \textbf{symbol tables} to be
      used by other parts of the compiler
    \item Be called lazily by the parser
  \end{itemize}

  \vspace*{1em}

  \begin{tikzpicture}[outer sep=1mm, node distance=8mm, auto]
    \footnotesize
    \node (source) {Source program};
    \node (lexer) [draw, right=of source] {Lexical analyzer};
    \node (table) [draw, below right=of lexer] {Symbol Table};
    \node (parser) [draw, right=of lexer:7] {Parser};

    \draw [->] (source) to (lexer);
    \draw [<->] (lexer) to (table);
    \draw [<->] (table) to (parser);

    \coordinate (lexer_u) at ($(lexer.north east)!.4!(lexer.south east)$);
    \coordinate (lexer_d) at ($(lexer.north east)!.6!(lexer.south east)$);
    \coordinate (parser_u) at ($(parser.north west)!.4!(parser.south west)$);
    \coordinate (parser_d) at ($(parser.north west)!.6!(parser.south west)$);

    \path[->]
      (lexer_u)     edge  node {\texttt{token}}         (parser_u)
      (parser_d)    edge  node {\texttt{next\_token()}} (lexer_d)
      ;

    \node[right =of parser] {} edge [<-, thin, densely dotted] (parser);
  \end{tikzpicture}
\end{frame}

% ---------------------------
\renewcommand{\sectiontitle}{Parsing}
\renewcommand{\currenttitle}{\sectiontitle}
\begin{frame}[fragile]{\currenttitle}
  Now that we have tokens, we need to make meaning of them by parsing them into
  a syntax tree \\[1em]

  To describe how we parse and what kind of tree we construct, we define the
  syntax of the language using a \textbf{context-free grammar}
\end{frame}

% ---------------------------
\renewcommand{\currenttitle}{Context-free grammars}
\begin{frame}[fragile]{\currenttitle}
  \vspace*{1em}
  Most commonly, we use \textbf{Extended Bakus-Naur form} notation \\[1.5em]

  EBNF for a small numeric expression calculator: \\[1em]

  \begin{lstlisting}[xleftmargin=2em]
expr      = binary_op
          | unary_op
          | num
          | "(", expr, ")"  ;
binary_op = expr, "+", expr
          | expr, "-", expr
          | expr, "*", expr
          | expr, "/", expr ;
unary_op  = "-", expr       ;
num       = integer
          | decimal   ;
  \end{lstlisting}
\end{frame}

% ---------------------------
\begin{frame}[fragile]{\currenttitle}
  Context-free grammars use sets of \textbf{production rules} that define the
  structure of the language:
  \begin{itemize}
    \item<+-> A rule contains a \textbf{nonterminal} symbol on the right and a
      sequence of both nonterminal and \textbf{terminal} symbols on the left
    \item<+-> Terminals are base units that can't be broken down into other symbols
      (they have no production rules of their own)
      \begin{itemize}
        \item Shown in quotes in the above EBNF
        \item In our case, tokens we get from the lexer
      \end{itemize}
  \end{itemize}
\end{frame}

% ---------------------------
\renewcommand{\currenttitle}{Parsing algorithms}
\begin{frame}[fragile]{\currenttitle}
  There are a number of parsing algorithms we can use to build our syntax tree:

  \begin{itemize}
    \item<2-> Top-down parsing
      \begin{itemize}
        \item LL
        \item Recursive descent
        \item Pratt parsing
      \end{itemize}
    \item<3-> Bottom-up parsing
      \begin{itemize}
        \item LR\footnote{Many subcategorizations depending on lookahead size
          and parsing table creation method}
          (SLR, LALR, canonical LR, GLR, etc.)
      \end{itemize}
  \end{itemize}

  \onslide<3->{We can of course use a mixture of these when appropriate}
\end{frame}

% ---------------------------
\begin{frame}[fragile]{\currenttitle}
  Different parsing methods can handle different \textbf{classes} of grammars:
  \onslide<+->{}

  \begin{itemize}
    \item<+-> Grammars parseable by LL parser are \textbf{LL grammars}
      \begin{itemize}
        \item Often produce lop-sided parse trees
      \end{itemize}
    \item<+-> Grammars parseable by LR parser are \textbf{LR grammars}
      \begin{itemize}
        \item Larger class of grammars than LL
      \end{itemize}
  \end{itemize}
\end{frame}

% ---------------------------
\begin{frame}[fragile]{\currenttitle}
  Textbooks traditionally recommend LR parsing because it can handle more
  complex grammars in more robust ways

  However, it is:

  \begin{itemize}
    \item<+-> Often tedious and lack of optimization can waste a lot of
      efficiency
    \item<+-> Too complex to create manually in practice
      \begin{itemize}
        \item We instead use \textbf{parser generators}
      \end{itemize}
  \end{itemize}

  \onslide<+->{Many modern compilers eschew parser generators in favor of
  handwritten top-down recursive descent parsers}
\end{frame}

% ---------------------------
\renewcommand{\sectiontitle}{Syntax translation}
\renewcommand{\currenttitle}{\sectiontitle}
\begin{frame}[fragile]{\currenttitle}
  During parsing, the syntax tree is constructed by creating tree nodes with
  metadata \textbf{attributes}

  We use \textbf{semantic rules} for each production rule to produce such a
  tree

  \begin{center}
    \begin{tabular}{l |l}
      Production                  & Semantic Rule                       \\
      \hline
      $E \rightarrow E_1 + E_2$   & $E$.val = $E_1$.val + $E_2$.val \\
      $E \rightarrow E_1 - E_2$   & $E$.val = $E_1$.val - $E_2$.val \\
    \end{tabular}
  \end{center}

  How this is implemented depends of course on the parsing method used
\end{frame}

% ---------------------------
\renewcommand{\currenttitle}{Syntax translation}
\begin{frame}[fragile]{\currenttitle}
  To model a syntax tree, we could define our tree nodes like so with an
  inheritance scheme:

  \begin{lstlisting}[language=C++]
class Expression {  ...  };

class BinOp extends Expression {
  BinOp(Op op, Expression e1, Expression e2) {  ...  }
};

class UnaryOp extends Expression {
  UnaryOp(Op op, Expression e) {  ...  }
}

enum class Op { Add, Sub, Neg };
  \end{lstlisting}
\end{frame}

% ---------------------------
\begin{frame}[fragile]{\currenttitle}
  We might have the following grammar and translation scheme: \\[1em]

  \begin{tabular}{l |l}
    Production & Translation Rule \\
    \hline
    $E \rightarrow E_1 + E_2$   & \texttt{$E$.node = BinOp(Op::Add, $E_1$.node, $E_2$.node)} \\
    $E \rightarrow E_1 - E_2$   & \texttt{$E$.node = BinOp(Op::Sub, $E_1$.node, $E_2$.node)} \\
    $E \rightarrow {-E_1}$      & \texttt{$E$.node = UnaryOp(Op::Neg, $E_1$.node)} \\
  \end{tabular}
\end{frame}

% ---------------------------
\begin{frame}[fragile]{\currenttitle}
  We might also modify the symbol table during parsing, just as with lexing:
  \\[1.5em]

  \begin{table}
    \centering
    \begin{tabular}{l |l}
      Production                    & Translation Rule \\
      \hline
      $S \rightarrow$ var $D = E ;$ & \texttt{table.insert(D.name);} \\
                                    & \texttt{$S$.node = Decl(D.name, E.val)}
    \end{tabular}
    \caption{Insert the new identifier to the symbol table when we encounter a
    variable declaration}
  \end{table}
\end{frame}
  
% ---------------------------
\renewcommand{\sectiontitle}{Semantic analysis}
\renewcommand{\currenttitle}{\sectiontitle}
\begin{frame}[fragile]{\currenttitle}
  The tree emitted by the parser can then be analyzed and transformed to check
  the for non-syntax program errors:

  \begin{itemize}
    \item Mismatched types
    \item Use of an undeclared identifier
    \item Invalid operations
    \item Other language-specific checking (e.g. lifetime analysis)
  \end{itemize}
\end{frame}
  
% ---------------------------
\renewcommand{\currenttitle}{Type analysis, checking, inference}
\begin{frame}[fragile]{\currenttitle}
  In many languages, types are statically checked at compile time; this would
  occur during the semantic analysis phase \\[2em]

  A classic and widely used method of checking and inferring types is the
  \textbf{Hindley-Milner type system}

  We can infer that the variables \texttt{a} and \texttt{x} are of type
  \texttt{u32} 

  \begin{lstlisting}[language=Rust, xleftmargin=5mm]
fn sum(a: u32, b: u32) -> u32 {
  return a + b;
}

let a = 5;
let x = sum(a, 10);
  \end{lstlisting}
\end{frame}
  
% ---------------------------
\renewcommand{\currenttitle}{Intermediate representations}
\begin{frame}[fragile]{\currenttitle}
  During analysis and optimization, it's usually necessary to use different
  representations (other than an abstract syntax tree) more conducive to
  different kinds of analysis.

  
\end{frame}
  
% ---------------------------
\renewcommand{\currenttitle}{}
\begin{frame}[fragile]{\currenttitle}
\end{frame}

% -----------------------------------------------

\end{document}
