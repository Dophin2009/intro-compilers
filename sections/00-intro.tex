% -----------------------------------------------
% chktex-file 44
\documentclass[../index.tex]{subfiles}
\begin{document}

% -------------------------------------

\renewcommand{\sectiontitle}{Before we begin}
\renewcommand{\currenttitle}{What we're going to be covering}
\begin{frame}{\currenttitle}
  
  A shallow review of:

  \begin{itemize}
    \item Requirements and objectives
      \begin{itemize}
        \item Different kinds of compilers
      \end{itemize}
    \item Organization of a basic compiler
      \begin{itemize}
        \item Lexical analyzers and regular expressions
        \item Parsing algorithms
        \item Intermediate representations
        \item Analysis techniques
        \item Code generation
      \end{itemize}
    \item Useful tools for creating compilers
  \end{itemize}
\end{frame}

% -------------------------------------
\renewcommand{\sectiontitle}{Requirements and objectives}
\section{\sectiontitle}

% ---------------------------
\renewcommand{\currenttitle}{What is a compiler?}
\begin{frame}[fragile]{\currenttitle}
  A compiler \textbf{translates source code from one language to another} \\[2em]

  We're typically referring to \textbf{the translation of a high-level
  programming language to executable low-level code}
\end{frame}

% ---------------------------
\begin{frame}[fragile]{\currenttitle}
  A compiler might convert a C++ program into machine code compatible with the
  computer's CPU: \\[2em]

  \chgrand[first=0,last=1]
  \newcommand{\randbit}{\rand\arabic{rand}}
  \begin{center}
    \begin{tikzpicture}
      \node (Code) [draw, inner sep=0.5em, outer sep=2] {
        \begin{lstlisting}[language=C++,basicstyle=\tiny\ttfamily]
#include <iostream>

int main() {
  std::cout << "Hello, World!" << std::endl;
}
        \end{lstlisting}
      };
    \node (Compiler) [draw, below=0.3cm of Code, outer sep=2] {Compiler};
    \node (Output) [draw, below=0.3cm of Compiler, inner sep=0.5em, outer sep=2, text width=4.5cm] {
        {
          \centering
          \tiny\ttfamily
          \noindent
          \foreach \n in {1,...,4}{
            \foreach \m in {1,...,40} {\randbit} \\
          }
        }
    };

    \draw[->] (Code) to (Compiler);
    \draw[->] (Compiler) to (Output);
    \end{tikzpicture}
  \end{center}

  \let\randbit\undefined
\end{frame}

% ---------------------------
\begin{frame}[fragile]{\currenttitle}
  \vspace*{1em}
  Another compiler might convert Typescript into legacy-compatible JavaScript: \\[2em]

  \begin{center}
    \begin{tikzpicture}
      \node (Code) [draw, inner sep=0.5em, outer sep=2] {
        \begin{lstlisting}[language=ES6,basicstyle=\tiny\ttfamily]
class Greeter {
  greeting: string;
  constructor(message: string) {
    this.greeting = message;
  }
  greet() {
    return "Hello, " + this.greeting;
  }
}
        \end{lstlisting}
      };
    \node (Compiler) [draw, below=0.3cm of Code, outer sep=2] {Compiler};
    \node (Output) [draw, right=1.4cm of Compiler, inner sep=0.5em, outer sep=2] {
        \begin{lstlisting}[language=ES6,basicstyle=\tiny\ttfamily]
var Greeter = (function() {
  function Greeter(message) {
    this.greeting = message;
  }
  Greeter.prototype.greet = function() {
    return "Hello, " + this.greeting;
  }
  return Greeter;
})();
        \end{lstlisting}
    };

    \draw[->] (Code) to (Compiler);
    \draw[->] (Compiler) to (Output);
    \end{tikzpicture}
  \end{center}
\end{frame}

% ---------------------------
\begin{frame}[fragile]{\currenttitle}
  Compilers can produce different outputs, depending on the purpose:

  \begin{itemize}
    \item<+-> Machine code
    \item<+-> Bytecode
    \item<+-> Assembly code
    \item<+-> Intermediate code forms (typically for internal use)
    \item<+-> Other high-level programming language code
    \item<+-> Etc.
  \end{itemize}
\end{frame}

% ---------------------------
\begin{frame}[fragile]{\currenttitle}
  Some examples:

  \begin{itemize}
    \item<+-> \texttt{javac} compiles Java source code into bytecode class
      files, which are executed by the JVM
    \item<+-> Clang can compile C/C++ into machine code, LLVM IR,
      various assembly forms, etc.
    \item<+-> CPython produces bytecode from Python source that is then
      executed
  \end{itemize}
\end{frame}

% ---------------------------
\begin{frame}[fragile]{\currenttitle}
  We can use different names for these different kinds of compilers:

  \begin{itemize}
    \item<+-> Source-to-source compilers or \textit{transpilers} transform
      high-level code into other high-level code
    \item<+-> Bytecode compilers produce low-level instructions to be executed
    \item<+-> JIT compilers defer compilation until runtime, and compile
      frequently run "hot" code paths
    \item<+-> Assemblers convert assembly language into machine code
  \end{itemize}
\end{frame}

% ---------------------------
\begin{frame}[fragile]{\currenttitle}
  In the context of native machine code, it's meaningful to make a distinction
  between compilers and \textbf{linkers}:

  \begin{itemize}
    \item<2-> A compiler produces \textbf{object files}
    \item<3-> A compile-time linker combines object files and static
      libraries into an executable
    \item<4-> A runtime linker will load dynamic libraries at runtime
  \end{itemize}

  \onslide<4->{
    If you've never written or compiled C/C++, these ideas may be foreign to you
  }
\end{frame}

% ---------------------------
\begin{frame}[fragile]{\currenttitle}
  \vspace*{1em}

  \newcommand{\outputt}[2]{
    \begin{tabular}{|l |l |}
      \hline
      \textbf{\small #1} & \cellcolor{black!20}\texttt{\scriptsize #2} \\
      \hline
    \end{tabular}
  }

  \begin{center}
    \begin{tikzpicture}[
        auto,
        node distance=0.8,
        tool/.style={draw, font=\bfseries, outer sep=1mm, preaction={fill, black!10}},
        output/.style={draw=none},
      ]
      \node (source)      [output]                        {\outputt{Source code}{*.cpp, etc.}};
      \node (compiler)    [tool, right=of source]         {Compiler};
      \node (object)      [output, below=of compiler]     {\outputt{Object files}{*.o, *.obj}};
      \node (linker)      [tool, below=of object]         {Linker};
      \node (executable)  [output, below=2 of linker]     {\outputt{Executable}{ \_ ,*.exe}};

      \node (lib)         [output, below left=of linker]  {\outputt{Library objects}{*.o, *.obj}};
      \node (os)          [output, left=of linker]        {\outputt{OS code}{}};

      \draw[->] (source)    to (compiler);
      \draw[->] (compiler)  to (object);
      \draw[->] (object)    to (linker);
      \draw[->] (lib)       to (linker);
      \draw[->] (os)        to (linker);
      \draw[->] (linker)    to (executable);
    \end{tikzpicture}
  \end{center}

\end{frame}

% ---------------------------
\renewcommand{\currenttitle}{Requirements}
\begin{frame}[fragile]{\currenttitle}
  A compiler developer is subject to stringent requirements \\[2em]

  A compiler must be:
  \begin{itemize}
    \item \textbf{Fast} \textendash{} We must be able to compile source
      programs in a reasonable amount of time
    \item \textbf{Correct} \textendash{} The resulting program must match
      the intent of the programmer and language specification
    \item \textbf{Optimized} \textendash{} The resulting program must be
      efficient and make good use of the target machine's capabilities
  \end{itemize}
\end{frame}

% -----------------------------------------------

\end{document}
