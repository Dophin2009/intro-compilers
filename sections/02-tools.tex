% -----------------------------------------------
% chktex-file 44
\documentclass[../index.tex]{subfiles}
\begin{document}

% -------------------------------------

\renewcommand{\sectiontitle}{I want to write a compiler\textemdash{what do?}}
\section{\sectiontitle}

% ---------------------------
\renewcommand{\currenttitle}{First, why?}
\begin{frame}{\currenttitle}
  Why might you want to create a compiler?

  \only<+->{}
  \begin{itemize}
    \item<+-> Writing a domain-specific language (DSL) such as a custom configuration or specification language
    \item<+-> Learn about useful data structures and algorithms
    \item<+-> Having a better understanding of compilers will make you a better programmer
    \item<+-> Software engineering exercise
    \item<+-> Etc.
  \end{itemize}
\end{frame}

% ---------------------------
\renewcommand{\currenttitle}{My first recommendation}
\begin{frame}{\currenttitle}
  Learn a lower-level language (such as C, C++, or Rust):
  
  \begin{itemize}
    \item Teach you a lot about memory management
    \item Expose you to systems programming concepts
  \end{itemize}
\end{frame}

% ---------------------------
\renewcommand{\currenttitle}{Learning materials}
\begin{frame}{\currenttitle}
  Follow a tutorial or book:

  \begin{itemize}
    \item Crafting Interpreters\footnote{\url{https://craftinginterpreters.com/}} (Java)
    \item LLVM Kaleidoscope language tutorial (C++)
    \item Dragon Book (\textit{Compilers: Principles, Techniques, and
      Tools})\footnote{College textbook material; heavy use of weird formal
                       language and mathematical notation}
    \item \textit{Modern Compiler Design}
  \end{itemize}
\end{frame}

% ---------------------------
\renewcommand{\currenttitle}{Useful tools}
\begin{frame}{\currenttitle}
  If you're not hell-bent on doing everything from scratch, use of pre-existing
  tools will help make your compiler better and easier to develop:

  \begin{itemize}
    \item Lexer generators
    \item Parser generators
    \item Analysis and optimization infrastructures
  \end{itemize}
\end{frame}

% ---------------------------
\renewcommand{\currenttitle}{Lexer generators}
\begin{frame}{\currenttitle}
  Instead of handwriting a lexer or implementing regular expression and state
  machines, you can use a tool that will generate lexers \\[1.5em]

  Typically:

  \begin{itemize}
    \item Input regular expressions and the tokens they resolve to
    \item Can embed code to process the text from the input
  \end{itemize}
\end{frame}

% ---------------------------
\begin{frame}[fragile]{\currenttitle}
  \vspace*{1em}
  Example for the lexer generator I wrote (Rust): \\[1em]

  \begin{lstlisting}[language=Rust]
lexer! {
    pub struct Lexer;
    pub fn stream;
    (text) -> Token, Token::Error;

    r"\s" => None,
    r"[0-9]+" => {
        let i = text.parse().unwrap();
        Some(Token::Integer(i))
    }
    r"[0-9]+(\.[0-9]+)?" => {
        let f = text.parse().unwrap();
        Some(Token::Float(f))
    }
}
  \end{lstlisting}
\end{frame}

% ---------------------------
\renewcommand{\currenttitle}{Parser generators}
\begin{frame}{\currenttitle}
  \textbf{Parser generators} will generate a parser for a grammar:

  \begin{itemize}
    \item Define production rules
    \item Embed code to serve as translation rules (to construct an AST,
      evalulate expressions, etc.)
    \item Most commonly LR parsers
  \end{itemize}
\end{frame}

% ---------------------------
\renewcommand{\currenttitle}{LLVM}
\begin{frame}{\currenttitle}
  \textbf{LLVM} is a compiler infrastructure that can provide the intermediate
  and back-end phases:

  \only<+->{}
  \begin{itemize}
    \item<+-> Defines its own SSA-based IR that it can perform all kinds of
      optimizations on
    \item<+-> Has back-ends implemented for many different target machines
    \item<+-> Many production compilers implemented with LLVM
      \begin{itemize}
        \item C\#
        \item Clang (C/C++)
        \item Haskell
        \item Rust
        \item Etc.
      \end{itemize}
  \end{itemize}
\end{frame}

% -----------------------------------------------

\end{document}
